% Todo: check glossaryentries

\newglossaryentry{Parsing}{
	name=Parsing,
	description={
		Parsing oder auch \emph{Syntaxanalyse} erzeugt aus einer Zeichenkette einen \gls{AST} aufgrund der Regeln einer Grammatik. Der Aufbau der Zeichenkette wird also nach den Regeln der Grammatik analysiert und die einzelnen Elemente in einen \gls{AST} überführt.
	}
}

\newglossaryentry{REST}{
	name=\textsc{Rest}, 
	description={
		\emph{Representational State Transfer} (deutsch: \enquote{Gegenständlicher Zustandstransfer}) ist ein Softwarearchitekturstil für Webanwendungen, welcher von Roy Fielding in seiner \printhref{http://www.ics.uci.edu/~fielding/pubs/dissertation/fielding_dissertation.pdf}{Dissertation} beschrieben wurde. Die Daten liegen dabei in eindeutig addressierbaren \emph{resources} vor. Die Interaktion basiert auf dem Austausch von \emph{representations} -- also ein Dokument was den aktuellen oder gewünschten Zustand einer resource beschreibt.
		Beispiel-URL für das Item \emph{84} aus dem Warenkorb \emph{42}:\\
		\texttt{http://api.spreadshirt.net/api/v1/baskets/84/item/42}
	}
}

\newglossaryentry{RESTful}{
	name=\textsc{Rest}ful,
	description={
		Als \emph{RESTful} bezeichnet man einen Webservice der den Prinzipien von REST entspricht
	},
	see=REST
}

\newglossaryentry{API}{
	name=\textsc{Api}, 
	description={
		\emph{Application Programming Interface} (deutsch: \enquote{Schnittstelle zur Anwendungsprogrammierung}) spezifiziert, wie Softwarekomponenten über diese Schnittstelle miteinander interagieren können
	}
}

\newglossaryentry{XML}{
	name=\textsc{Xml},
	description={
		\emph{Extensible Markup Language} (deutsch: \enquote{erweiterbare Auszeichnungssprache}) ist ein mensch- und maschinenlesbares Format für Codierung und Austausch von Daten, \printhref{http://www.w3.org/TR/REC-xml}{spezifiziert vom W3C}
	}
}

\newglossaryentry{JSON}{
	name=\textsc{Json},
	description={
		\emph{JavaScript Object Notation} ist ein Mensch- und Maschinenlesbares Format zu Codierung und Austausch von Daten. Es bietet im Gegensatz zu XML keine Erweiterbarkeit und Unterstützung für Namesräume, ist aber kompakter und einfacher zu parsen
	}, 
	see=XML
}

\newglossaryentry{XSD}{
	name=\textsc{Xsd},
	description={
		\emph{XML Schema Description}, auch nur \emph{XML Schema} ist eine Schemabeschreibungssprache und enthält Regeln für den Aufbau und zum Validieren einer XML-Datei. Die Beschreibung ist selbst wieder eine gültige XML-Datei
	},
	see=XML
}

%\newglossaryentry{RelaxNG}{
%	name=RelaxNG,
%	description={
%		\emph{Regular Language Description for XML New Generation} ist ebenso wie \emph{XSD} eine Schemabeschreibungssprache, bietet aber zwei Syntaxformen, eine XML basierte und eine kompaktere eigene Syntax
%	}, 
%	see=XSD
%

\newglossaryentry{WADL}{
	name=\textsc{Wadl},
	description={
		\emph{Web Application Description Language} ist eine maschinenlesbare Beschreibung einer HTTP-basierten Webanwendung
	},
	see=XML
}

\newglossaryentry{Polyglot}{
	name=Polyglot,
	description={
		\emph{mehrsprachig}
	}
}

\newglossaryentry{Metaprogramming}{
	name=Metaprogramming,
	description={
		beschreibt das erstellen von Programmen welche sich selbst, oder andere Programme, modifizieren oder die einen Teil des Kompilierungsschrittes übernehmen (bspw. der C-Präprozessor)
	}
}

\newglossaryentry{DSL}{
	name=\textsc{Dsl},
	description={
		\emph{Domain Specific Language} (deutsch: \enquote{Domänenspezifische Sprache}) ist eine Programmiersprache die nur auf eine bestimmte Domäne oder auch Problembereich optimiert ist.
	}
}

\newglossaryentry{template-engine}{
	name=Template-Engine,
	description={
		Eine \emph{Template-Engine} ersetzt markierte Bereiche in einer Template-Datei (i. Allg. Textdateien) nach vorgegebenen Regeln
	}
}

\newglossaryentry{MIME}{
	name=\textsc{Mime},
	description={
		\emph{Multipurpose Internet Mail Extensions} dienen zu Deklaration von Inhalten (Typ des Inhalts) in verschiedenen Internetprotokollen
	}
}

\newglossaryentry{URI}{
	name=\textsc{Uri},
	description={
		\emph{Unified Resource Identifier} ist ein Folge von Zeichen, die einen Name oder eine Web-Ressource identifiziert
	},
	plural=\textsc{Uri}s
}

\newglossaryentry{URL}{
	name=\textsc{Url},
	description={
		\emph{Unified Resource Locator} sind eine Untermenge der \emph{URIs}. Der Unterschied besteht in der expliziten Angabe des Zugrissmechanismus und des Ortes (\enquote{Location}) durch \emph{URLs}, bspw. \texttt{http} oder \texttt{ftp}
	},
	plural=\textsc{Url}s,
	see=URI
}

\newglossaryentry{DTD}{
	name=\textsc{Dtd},
	description={
		\emph{Document Type Definition}, manchmal auch \emph{Data Type Definition} ist eine Menge von Angaben, die einen Dokumenttyp beschreiben. Es werden konkret Element- und Attributtypen, Entitäten und deren Struktur beschrieben. Die bekanntesten Schemasprachen für XML-Dokumente sind XSD und RelaxNG
	},
	see=XSD
}

% http://en.wikipedia.org/wiki/Model-driven_architecture
\newglossaryentry{MDA}{
	name=\textsc{Mda},
	description={
		\emph{Model Driven Architecture} ist ein modell-getriebener Softwareentwicklungsansatz. Das zu modellierende System wird hierbei durch ein plattformunabhängiges Modell beschrieben mittels einer \gls{DSL} beschrieben. Dieses Modell wird dann durch einen Generator in ein plattformspezifisches Modell, meist in einer \gls{GPL} übersetzt
	}
}

% Wikipedia
\newglossaryentry{MDSD}{
	name=\textsc{Mdsd},
	description={
		\emph{Model Driven Software Development}, auch \emph{Model Driven Engineering} ist eine Softwareentwicklungsmethode welche ihren Fokus auf das erzeugen und nutzen von Domänen-Modellen, anstelle der algorithmischen Konzepte, legt
 	}	
}

\newglossaryentry{MDE}{
	name=\textsc{Mde},
	description={
		\emph{Model Driven Engineering}
	},
	see=MDSD
}

\newglossaryentry{AST}{
	name=Abstract Syntax Tree,
	description={
		Ein \emph{Abstrakter Syntaxbaum} ist die Baumdarstellung einer abstrakten Syntaktischen Struktur von Quellcode einer Programmiersprache. Jeder Knoten des Baumes kennzeichnet ein Konstrukt des Quellcodes. Der \emph{AST} stellt für gewöhnlich nicht alle Details des Quelltextes dar, bspw. formatierende Element wie etwa Klammern werden häufig weggelassen
	}
}

\newglossaryentry{GPL}{
	name=General Purpose Language,
	description={
		Eine \emph{General Purpose Language} bezeichnet eine Programmiersprache welche für den Einsatz in den verschiedensten Anwendungsbereichen verwendet kann, im Gegensatz zu einer \gls{DSL}, welche nur auf einen speziellen Bereich beschränkt ist
	}
}

\newglossaryentry{closure}{
	plural=Closures,
	name=Closure,
	description={
		Eine \emph{Closure} ist eine Funktion, welche die besondere Eigenschaft besitzt auf Variablen aus ihrem Entstehungskontext zugreifen zu können. Die Funktion wird meist in einer Variablen gespeichert um den Zugriff darauf zu sichern
	}
}