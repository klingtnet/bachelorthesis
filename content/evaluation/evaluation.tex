\chapter{Evaluierung}
\label{chap:evaluation}

\chapterQuote{There are two ways to write error-free programs; only the third one works}{\citeauthor{alanPerlis}}{\citeyear{alanPerlis}}{\cite{alanPerlis}}

In diesem Kapitel soll der Generator und die erzeugte Bibliothek gegenüber den Anforderungen evaluiert werden. Dies geschieht für die Nutzbarkeit (\cref{sec:usability}) anhand eines einfachen Anwendungsbeispieles.

\section{PHP-Zielsprachenmodell}
\label{sec:php_target_language_model}

Das PHP-Zielsprachenmodell ist die Implementierung der Schnittstellen die durch das Sprachenmodell vorgegeben sind.
Eine Gegenüberstellung eines Ausschnittes einer generierten Klassendatei mit ihrer Repräsentation im \gls{AST} der aus Elementen des Sprachenmodells gebildet wird, zeigen \Cref{fig:modelRepresentationOfBatchDTO} und \Cref{lst:batchDTO}.

\begin{sidewaysfigure}
    \centering
    \resizebox{\textwidth}{!}{
      \begin{tikzpicture}[
    level distance=1.2cm,
    level 1/.style={sibling distance=5cm}
    ]
    \node (root) [classNode] {ClassFile} 
    [edge from parent fork down]
    child [sibling distance = 3cm] { node [classNode] {Import}
        child { node [classNode] {Literal}
            child { node [classNode] {PrimitiveType}
                child { node [stringNode] {'Dto.php'} }
            }
        }
    }
    child { node [classNode] {Import}
        child { node [classNode] {Literal}
            child { node [classNode] {PrimitiveType}
                child { node [stringNode] {'OperationDTO.php'} }
            }
        }
    }
    child { child { child { 
    child { node [classNode] {CommonClass}
        child { node [stringNode] {'BatchDTO'} }
        child { node [classNode] {AssignmentStatement}
            child [sibling distance = 3cm] { node [classNode] {Variable} 
                child { node [stringNode] {'operations'} }
                child { node [classNode] {Modifiers} 
                    child { node [stringNode] {'private'} }
                }
            }
            child [sibling distance = 3cm] { node [classNode] {Operator}
                child { node [stringNode] {'='} }
            }
            child [sibling distance = 3cm] { node [classNode] {MethodInvocation} 
                child { node [stringNode] {'array'} }
            }
        }
        child { child { child {
        child [sibling distance = 3.5cm] { node [classNode] {DefinitionStatement} 
            child { node [stringNode] {'fromXML'} }
            child { node [classNode] {Modifiers}
                child { node [stringNode] {'public'} }
                child { node [stringNode] {'static'} }
            }
            child {
            child { node [classNode] {Identifier}
                child { node [classNode] {Comment} 
                    child { node [stringNode] {'SimpleXMLElement'} }
                }
                child { node [stringNode] {'xml'} }
            }
            }
            child { child { 
            child { node [classNode] {Block}
                child { node [classNode] {AssignmentStatement}
                                   child { node [classNode] {Identifier}
                        child { node [stringNode] {'operations'} }
                    }                    
                    child { node [classNode] {'Operator'} 
                        child { node [stringNode] {'::'} }
                    }
                    child { node [classNode] {'MethodInvocation'} 
                        child { node [stringNode] {'fromXML'} }
                        child { node [classNode] {SimpleExpression} 
                            child { node [classNode] {Identifier}
                                child { node [classNode] {Comment}
                                    child { node [stringNode] {'SimpleXMLElement'} }
                                }
                                child { node [stringNode] {'xml'} }
                            }
                            child { node [classNode] {Operator}
                                child { node [stringNode] {'->'} }
                            }
                            child { node [classNode] {Identifier} 
                                child { node [stringNode] {'operations'} }
                            }
                        }
                    }
                }
            }
            }}
        }
        }}}
    }}}
    }
    ;
\end{tikzpicture}

    }
    \caption{Darstellung von BatchDTO aus \Cref{lst:batchDTO} im Sprachenmodell [\textbf{Klasse}, \emph{Zeichenkette}]}
    \label{fig:modelRepresentationOfBatchDTO}
\end{sidewaysfigure}

\subsection{Implementierung einer Zielsprache}
\label{sec:target_language_implementation}

Um eine neue Zielsprache für den Generator zu implementieren ist es notwendig alle Interfaces des Sprachenmodells (\cref{sec:language_model}) zu implementieren. Methoden zum Erhalt von Schlüsselwörtern, Sichtbarkeitsmodifikatoren und Symbolen (z.B. Inkrement, Dereferenzoperator, usw.) sind in den Interfaces \emph{Keywords, Modifiers} und \emph{Symbols} deklariert. Die Syntax der Zielsprache, sowie der verwendete Code-Style werden durch die Implementierung des Interfaces \emph{LanguageVisitor} festgelegt.

Neben den Interfaces des Sprachenmodells müssen auch die statischen Klassen in der Zielsprache implementiert werden.

\begin{minipage}{\textwidth}
\begin{lstlisting}[
    language=PHP,
    caption=Ausschnitt der generierten Datenklasse BatchDTO,
    label=lst:batchDTO
]
<?php
   require_once('Dto.php');
   require_once('OperationDTO.php');

   class BatchDTO
   {
      private $operations = array(); // operationDTO 
      ...
      public static function fromXML(
            /* SimpleXMLElement */ $xml
         )
      {
         $operations = OperationDTO::fromXML(/* SimpleXMLElement */ $xml->operations)
;          ...
      }
      ...
   }
?>
\end{lstlisting}
\end{minipage}

\subsection{Codestyle/Lesbarkeit}
\label{se:code_style_readability}

Um den erzeugten Code möglichst lesbar zu gestalten, wurden folgende Konventionen im \emph{LanguageVisitor} implementiert:
\begin{compactitem}
  \item erhöhen der Einrückungstiefe innerhalb jedes Codeblockes
  \item Zeilenumbruch vor jedem Block-Begrenzer (\texttt{\{, \}})
  \item Zeilenumbruch vor jedem Parameter in einer Funktionsdefinition (\Cref{lst:generatedRessourceClass}, Zeile~\ref{lst:methodParameters} und folgende)
\end{compactitem}

\section{Typsicherheit}
\label{sec:type_safety}

\textsc{Php} bietet durch seine dynamische Typisierung leider keine Typprüfung zur Compilezeit. Da dem Generator die erwarteten Typen bekannt sind, werden die Typnamen an entsprechender Stelle im Code durch ein Block-Kommentar eingefügt (wie u.a. in \cref{lst:generatedRessourceClass} und \Cref{lst:generatedDataClass} zu sehen). Somit ist zumindest eine manuelle Prüfung der zu übergebenden Typen für den Nutzer der Bibliothek möglich.

Durch die Implementierung speziell formatierter Kommentare --- beispielsweise \texttt{/** @var [typename] */} in IntelliJ \textsc{Idea} --- kann eine Typprüfung durch die \textsc{Ide} ermöglicht werden. Um sich nicht auf eine spezielle \textsc{Ide} festlegen zu müssen, werden derzeit normale Block-Kommentare an dieser Stelle erzeugt.

\section{Generierung von Tests für die erzeugte Bibliothek}
\label{sec:test_coverage}

Die Generierung von Tests für die erstellte Bibliothek ist derzeit noch nicht implementiert. Ein Testszenario für die erzeugten Datenklassen wäre die Prüfung auf Gleichheit vor und nach der Serialisierung. Zur Überprüfung der Ressourcenklassen werden erfolgreiche Tests der Datenklassen vorausgesetzt, da die Testdaten aus den Datenklassen erstellt werden. Die erfolgreiche Ausführung der Methoden einer Ressourcenklassen lässt sich über deren Statuscode überprüfen.

Eine andere Vorgehensweise zur Prüfung der erzeugten Bibliothek wäre die manuelle Erstellung von Tests basierend auf typischen Anwendungsfällen, wie dem Anlegen neuer Designs oder dem Erstellen neuer Produkte. Besser wäre dieses Vorgehen zusätzlich zu automatisch erzeugten Tests durchzuführen um eine möglichst hohe Abdeckung zu erreichen.

\section{Metriken}
\label{sec:codemetrics}

Die Erzeugung der Client-Bibliothek erfolgt nur bei Änderungen an der Spreadshirt-\gls{API}, weshalb keine besonderen Anforderungen an die Dauer des Generierungsvorganges gestellt wurden. Um aus der \citetitle{WADL} \cite{WADL} die Bibliothek zu generieren, benötigt der Codegenerator:
\begin{compactitem}
  \item \emph{23.1s}, falls die \gls{API}-Beschreibung vom Server generiert werden muss.
  \item \emph{1.25s}, falls die \gls{API}-Beschreibung gecached ist. 
\end{compactitem}

Die Werte wurden über drei Generierungsvorgänge gemittelt und auf folgenden System erfasst:
\begin{compactitem}
  \item Intel\textsuperscript{\textregistered{}} Core\texttrademark{} i5-2520M CPU
  \item 8GB DDR3 RAM
  \item 256GB Samsung\textsuperscript{\textregistered{}} 830 SSD (ext4)
  \item Linux Kernel 3.8.0-31-generic (x64)
  \item Oracle\textsuperscript{\textregistered{}} Java\texttrademark{} SE Runtime Environment (build 1.7.0\_40-b43)
\end{compactitem}

Der Großteil der Zeit wird für den Verbindungsaufbau mit dem Server und dem anschließenden Transfer der \gls{API}-Beschreibung verbraucht.

\Cref{tab:code_metrics} zeigt den Umfang, der aus der \gls{API}-Beschreibung mit Stand vom 17. September 2013, erzeugten Bibliothek.
Das Kommando zur Ermittlung der Werte aus \cref{tab:code_metrics} lautet:\\
\texttt{find data/ resources/ -type f -name *.php -exec wc {} +}

\begin{table}
    \begin{longtable}{r r r r}
        \toprule
        \rowcolor{lightgray}
                          & \textbf{Dateien}  & \textbf{Zeilen}     & \textbf{Zeichen}\\
        \midrule
        Datenklassen      & 192               & 17455               & 38550\\
        Ressourcenklassen & 97                & 3730                & 11079\\        
        \midrule
        $\sum$            & 289               & 21185               & 49629\\
        \bottomrule
        \caption{Metriken zur Menge des generierten Codes (ohne statische Klassen)}
        \label{tab:code_metrics}
    \end{longtable} 
\end{table}

Jede erzeugte Datei erhält dabei eine Klasse. Die Gesamtheit aller in \cite{WADL} beschriebenen Ressourcen entsprechen der Anzahl der generierten Ressourcenklassen (97). Somit ist die Anforderung erfüllt alle Methoden aus der \gls{API}-Beschreibung zu generieren.

\section{Leistungsbewertung}
\label{sec:performance_measurement}

% todo: vollständige Generierung aller Methoden aus der \gls{API}-Beschreibung

Im folgenden soll eine typischer Arbeitsablauf mit der Client-Bibliothek gezeigt werden. Der Client sollte zum Einstieg ein Objekt der Datenklasse \textbf{LoginDTO} anlegen \ding{202} und seine Login-Informationen eintragen, dies wird benötigt um über die Ressourcenklasse \textbf{Sessions} auf der gleichnamigen \gls{API}-Ressource eine neue Sitzung (\enquote{Session}) anzulegen.
Die \emph{Response} der Session-Ressource enthält eine \gls{URI} auf die Ressource der angelegten Sitzung. Diese \gls{URI} enthält die \emph{SessionID} welche durch \ding{204} aus der \gls{URI} extrahiert wird.

Nun sind alle nötigen Informationen vorhanden um ein Objekt der Klasse \textbf{ApiUser} anzulegen \ding{205}, welches die Authentifizierungsinformationen für gesicherte Ressourcen der Spreadshirt-\gls{API} enthält.

Nun kann über die Bibliotheksklassen auch auf geschützte \gls{API}-Ressourcen zugegriffen werden, in diesem Fall werden beispielsweise die Produkte eines bestimmten Users abgefragt. Entegegen den vorherigen Aufrufen von Methoden der Ressourceklassen wird ein Parameter-Array erzeugt \ding{206} um den \emph{mediaType} der Response festzulegen.
Vor dem Zugriff auf die Ressource muss noch die entsprechende Ressourcenklasse instanziiert werden \ding{207}.

\begin{minipage}{\textwidth}
\begin{lstlisting}[
    language=PHP,
    caption=Beispiel für eine Interaktion mit der Spreadshirt-API über die generierte Client-Bibliothek (Authentifizierungsinformationen wurden anonymisiert),
    label=lst:giveMeALabel
]
<?php

require_once("data/LoginDTO.php");
require_once("Static/apiUser.php");
require_once("resources/UsersUserId.php");
require_once("resources/UsersUserIdProducts.php");
require_once("resources/Sessions.php");

$loginDTO = new LoginDTO(); //@\ding{202}@//
$loginDTO->setUsername("username");
$loginDTO->setPassword("password");
$resource = new Sessions();
$session = $resource->POST(null, $loginDTO); //@\ding{203}@//
$sessionId = preg_replace('/.*\//',"",$session['header']['Location']); //@\ding{204}@//

$apiUser = new ApiUser("userId", "apiKey", "secret", $sessionId); //@\ding{205}@//

$parameters = array("mediaType"=>"json"); //@\ding{206}@//
$resource = new UsersUserIdProducts($userId); //@\ding{207}@//
$products = $resource->GET($parameters, $apiUser); //@\ding{208}@//

?>
\end{lstlisting}
\end{minipage}


\section{Nutzbarkeit}
\label{sec:usability}

Quellcodedateien die Bibliotheksfunktionen nutzen müssen im Wurzelverzeichnis dieser Bibliothek liegen. Dies ist damit begründet das der \textsc{Php}-Interpreter bei Importanweisungen mit relativen Pfaden, den aktuellen Pfad der ausgeführten Datei als Basis nimmt.

Der Aufruf des Generators erfolgt über die Kommandozeile. Als einziges Argument wird die \gls{URL} zur \gls{WADL}-Datei erwartet. Beispiel:\\ 
{\footnotesize \texttt{java api-client-libs.jar http://api.spreadshirt.net/api/v1/metaData/api.wadl}}\\
Der Ausgabepfad der erzeugten Bibliothek ist relativ zum Aufrufpfad des Generators: \texttt{./generated/php}.

Derzeit ist die Bibliothek noch eingeschränkt nutzbar, da die De-/Serialisier von strukturierten Typen noch nicht fehlerfrei generiert werden. Die Informationen, die nötig sind um Datenklassen verlustfrei zu serialisieren und wieder zu deserialisieren sind im Schema-Modell (\cref{sec:schema_model}) vorhanden. Deshalb muss der Algorithmus im Codegenerator zur Erzeugung dieser Methoden überarbeitet werden.

