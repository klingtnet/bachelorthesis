\section{Client-Bibliothek}
\label{sec:client_library}

Die generierte Client-Bibliothek lässt sich in 2 verschiedene Arten von Klassen gliedern:
\begin{compactenum}
    \item die Elemente und Typen aus der \gls{XML}-Schemabeschreibung repräsentieren (\emph{Datenklassen});
    \item die \gls{API}-Ressourcen und deren Methoden abbilden (\emph{Ressourcenklassen}).
\end{compactenum}

Zusätzlich wurden für Aufgaben die keiner explizite Generierung bedürfen, wie \gls{HTTP}-Methodenaufrufe, manuell Klassen mit statischen Methoden erstellt. Dem Generator werden die Klassen als Abhängigkeiten im Eingabemodell bekannt gegeben und an enstprechender Stelle durch ihn als \emph{Import}-Anweisungen in der Bibliothek eingefügt. Keine explizite Generierung wird benötigt wenn der zu erzeugende Code keine oder sehr wenig variable Bestandteile enthält.

\subsection{Datenklassen}
\label{sec:dataclasses}

Die Datenklassen sind die zielsprachenabhängige Repräsentation der Elemente und Typen aus der \gls{XML}-Schemabeschreibung. 

Der Name der Klasse entspricht dabei der Bezeichnung des Typs oder des Elements. Die Variablen einer solchen Klasse sind die Attribute und Elemente aus der Schemabeschreibung des Typs. \textsc{Php} bietet keine Enumeratoren, deshalb werden die einzelnen Enum-Werte als statische Variablen vom Typ \texttt{string} generiert. Für alle Variablen werden außerdem Getter- und Setter-Methoden durch den Codegenerator erzeugt.

Konstruktoren zur Erzeugung von Objekten aus den Datenklassen werden ebenfalls vom Generator erstellt. Dabei werden die \emph{Häufigkeitsindikatoren} aus der Schemabeschreibung berücksichtigt. Bei einem \texttt{minOccurs}-Wert größer eins wird das Element zu den Konstruktor-Parametern hinzugefügt. Somit ist sichergestellt, dass notwendigen Angaben auch Werte zugeordnet werden.

Methoden zur De-/Serialisierung (\cref{sec:serialiser}) in eine der beiden von der Spreadshirt-\gls{API} unterstützten Dokumentbeschreibungsformate (\gls{JSON}, \gls{XML}) sind ebenfalls Bestandteil einer Datenklasse.

\Cref{lst:generatedDataClass} zeigt einen gekürzten Ausschnitt der generierten Datenklasse zum Element \emph{Point} aus der Schemabeschreibung der Spreadshirt-\gls{API}.

\subsection{Ressourcenklassen}
\label{sec:ressourceclasses}

Ressourcenklassen sind die zielsprachenabhängige Abbildung der Ressourcenbeschreibungen aus WADL-Datei der Spreadshirt-\gls{API}.

Eine Ressourcenklasse beinhaltet:
\begin{compactitem}
    \item[\ding{202}] ein Feld, welches die Basis-\gls{URL} der API beinhaltet;
    \item[\ding{203}] ein Feld in welches die komplette \gls{URL}, inklusive der ersetzten Template-Parameter (\cref{sec:rest_model}), der Ressource erhält;
    \item[\ding{204}] eine Menge von Feldern, die jeweils einem Template-Parameter zugeordnet sind;
    \item[\ding{205}] einen Konstruktor, dessen Argumente den Template-Parametern entsprechen und der aus diesen und der Basis-\gls{URL} die Ressourcen-\gls{URL} erstellt;
    \item[\ding{206}] Abbildungen der Methoden aus der Ressourcenbeschreibung. Methodenparameter, die zur Authentifizierung an der \gls{API} notwendig sind, werden durch einen Parameter der Klasse \emph{ApiUser} (\cref{sec:api_auth}) substituiert \ding{207}.
\end{compactitem}

\Cref{lst:generatedRessourceClass} beinhaltet die generierte Klasse zur Ressource \texttt{users/{userId}/products} der Spreadshirt-\gls{API}.

\begin{lstlisting}[
    language=PHP,
    caption=Point-Klasse als (gekürztes) Beispiel für eine generierte Datenklasse,
    label=lst:generatedDataClass
]
<?php
   require_once('Unit.php');

   class Point
   {
      private $unit; // unit 
      private $y; // double 
      private $x; // double 

      function __construct(
            /* double */ $y,
            /* double */ $x
         )
      {
         $this->y = $y;
         $this->x = $x;
      }

      public function setUnit(
            /* unit */ $unit
         )
      {
         $this->unit = $unit;
      }

      public function toJSON()
      {
         $json = json_decode(/* Point */ $this);
         return $json;
      }

      public function toXML()
      {
         $xml =  new SimpleXMLElement(/* Point */ '<login xmlns="http://api.spreadshirt.net"/>');
         $xml->addChild(/* string */ 'unit',/* unit */ $this->unit);
         $xml->addChild(/* string */ 'y',/* double */ $this->y);
         $xml->addChild(/* string */ 'x',/* double */ $this->x);
         return $xml->asXML();
      }

      public static function fromXML(
            /* SimpleXMLElement */ $xml
         )
      {
         $unit = Unit::fromXML(/* SimpleXMLElement */ $xml->unit);
         $y = $xml->y;
         $x = $xml->x;
         $Point =  new Point(/* double */ $y,/* double */ $x);
         $Point->setUnit(/* unit */ $unit);
         return $Point;
      }

      ...

      public function getX()
      {
         return $x = $this->x;
      }
   }
?>
\end{lstlisting}

\newpage

\begin{minipage}{\textwidth}
\begin{lstlisting}[
    language=PHP,
    caption=Klasse zur Ressource \texttt{users/{userId}/products} als Beispiel für eine Ressourcenklasse,
    label=lst:generatedRessourceClass
]
<?php
   require_once('Static/methods.php'); //@\label{lst:importMethods}@//
   require_once('Static/apiUser.php'); //@\label{lst:importApiUser}@//

   /* Create or list products for user. */
   class UsersUserIdProducts
   {
      private $baseUrl = 'http://192.168.13.10:8080/api/v1/'; // string //@\ding{202}@//
      public $userId; // string //@\ding{204}@//
      private $resourceUrl = ''; // string //@\ding{203}@//

      /*  */
      public function POST( //@\ding{206}@// //@\label{lst:methodParameters}@//
            /* array */ $parameters, 
            /* ApiUser */ $apiUser,
            /* ProductDTO */ $productDTO
         )
      {
         $auth = $apiUser->getAuthentificationHeader(/* string */ 'POST',/* string */ $this->resourceUrl);
         return Methods::post(/* string */ $this->resourceUrl,/* string */ $auth,/* array */ $parameters,/* ProductDTO */ $productDTO); //@\label{lst:staticMethodPost}@//
      }

      /* Sample Url is:  http://... */
      public function GET( //@\ding{206}@//
            /* array */ $parameters,
            /* ApiUser */ $apiUser //@\ding{207}@//
         )
      {
         $auth = $apiUser->getAuthentificationHeader(/* string */ 'GET',/* string */ $this->resourceUrl);
         return Methods::get(/* string */ $this->resourceUrl,/* string */ $auth,/* array */ $parameters); //@\label{lst:staticMethodGet}@//
      }

      function __construct( //@\ding{205}@//
            /* string */ $userId
         )
      {
         $this->userId = $userId;
         $this->resourceUrl = $this->baseUrl . 'users' . '/' . $userId . 'products';
      }
   }
?>
\end{lstlisting}
\end{minipage}

\subsection{De-/Serialisierer}
\label{sec:serialiser}
% todo: transportunabhängig als Schlagwort entfernen?

Um \emph{Ressourcen-Repräsentationen} (\cref{sec:representation}) mit der Spreadshirt-\gls{API} transportunabhängig austauschen zu können, müssen die strukturierten Datenklassen \emph{serialisiert} werden. In umgekehrter Richtung müssen \emph{Repräsentation} aus der \gls{API} deserialisiert --- also wieder in eine Datenklasse transformiert --- werden.

Die Datenklassen-Methoden zur Serialiserung und Deserialiserung besitzen einheitliche Bezeichner, nach dem Schema \texttt{toXML}, \texttt{toJSON}, respektive \texttt{fromXML}, \texttt{fromJSON}. Die Deserialisierer-Methoden sind \emph{statisch} um das unnötige Instanziieren einer Datenklasse zu vermeiden, nur um ihre Klassendarstellung aus der serialisierten Form zu erhalten.

Beispiele für beide Arten finden sich in \Cref{lst:generatedDataClass}.

\subsection{Statische Klassen}
\label{sec:static_classes}

\emph{Statische Klassen} bedeutet in dem Codegenerierungskontext, dass diese \emph{manuell} erstellt wurden. Die statischen Klassen enthalten Code der von anderen Klassen gemeinsam genutzt wird und keine variablen Bestandteile enthält. 

Prinzipiell werden zwei Probleme durch statische Klassen vermindert:
\begin{compactenum}
  \item unnötige Vergrößerung des generierten Codes
  \item Vermeidung des Aufwands zur Implementierung der statischen Inhalte in dem Eingabemodell des Generators
\end{compactenum}

Die generierte Client-Bibliothek beinhaltet dabei zwei dieser Klassen:
\begin{compactenum}
  \item zur Kommunikation mit der \gls{API} über \gls{HTTP}-Methoden
  \item zur Kapselung von Authentifizierungsinformationen
\end{compactenum}

Den Import beider Klassen zeigt \Cref{lst:generatedRessourceClass} in Zeile~\ref{lst:importMethods} und~\ref{lst:importApiUser}.

\subsubsection{Nutzung der HTTP-Methoden}
\label{sec:staticMethodsClass}

Um die generierten Ressourcenklassen nicht unnötig zu vergrößern wurde der \emph{einheitliche} Vorgang zum Aufruf der \gls{HTTP}-Methoden in eine manuell erstellte Klasse ausgelagert.

\Cref{lst:generatedRessourceClass} zeigt in Zeile~\ref{lst:staticMethodPost} und~\ref{lst:staticMethodGet} den Aufruf zweier solcher Methoden aus einer Ressourcenklasse.

\subsubsection{API Authentifizierung}
\label{sec:api_auth}

In der Spreadshirt-\gls{API} sind geschützte und ungeschützte Ressourcen vorhanden. 

Das zur Authentifizierung eines \gls{API}-Nutzers verwendete Protokoll \emph{SprdAuth} basiert auf \enquote{\gls{HTTP}s Authorization Request Header} sowie dem \enquote{\textsc{Www}-Authentificate Response Header} \cite{apiSecurity}.

Die Spreadshirt-\gls{API} unterstüzt die Übergabe der nötigen Autorisierungsparameter als Teil der \gls{URI}-Query oder in Form des \emph{Authorization-Header}. Die erzeugte Client-Bibliothek beschränkt sich auf die Nutzung des \emph{Authorization-Headers}, dieser besitzt folgenden Aufbau:

\begin{lstlisting}[
    language=JavaScript,
    caption=Aufbau des Spreadshirt Authentification Header,
    label=lst:authHeader
]
Authorization:  SprdAuth apiKey="<apikey>", //@\ding{202}@// 
                data="<method> <url> <time>", 
                sig="SHA1(<method> <url> <time> <secret> //@\ding{203}@// )", 
                sessionId="<sessionId>" //@\ding{204}@//
\end{lstlisting}

Die Klasse \classname{ApiUser} kapselt die Daten, die zur Autorisierung an der \gls{API} nötig sind und stellt eine Methode bereit, die dem Nutzer das Erstellen des Authorization-Headers erspart. Der Konstruktor der ApiUser-Klasse erwartet dabei die Angabe der folgenden Parameter:

Alle Methoden, die eine Autorisation benötigen erhalten vom Codegenerator einen Parameter vom Typ \emph{ApiUser} wie \ding{206} in \Cref{lst:generatedRessourceClass} zeigt.

\begin{compactitem}
    \item \emph{UserID}, die Indentifikationsnummer des Spreadshirt-Nutzers
    \item \emph{ApiKey} und \emph{Secret} (\ding{202}, \ding{203}), diese Informationen erhält der Nutzer, wenn er sich als \gls{API}-User bei Spreadshirt registriert
    \item \emph{SessionID} (\ding{204}), die SessionID ist in der Response der POST-Methode auf Ressource \texttt{sessions} enthalten.
\end{compactitem}
