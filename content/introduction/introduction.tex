\chapter{Einführung}

\chapterQuote{Essentially, all models are wrong, but some are useful.}{George E. P. Box, Norman R. Draper}{1987}{Empirical Model-Building and Response Surfaces. p. 424}

\begin{figure}[b]
    \centering
    \resizebox{\textwidth}{!}{
        \begin{tikzpicture}[
    	node distance=12mm and 8mm,
    	every node/.style={font=\scriptsize}
    ]
    % Blocks
    \node(abstract)[greyBlock, double copy shadow]{abstrakte\\API Beschreibung};
    \node(parser)[greyBlock, right=of abstract]{Parser};
    \node(generator)[greyBlock, right=of parser]{Codegenerator\\\emph{Java}};
    \node(infrastructurecode)[greyBlock, above=of generator]{Infrastrukturcode};
    \node(templates)[greyBlock, double copy shadow, below=of generator]{Templates};        
	\node(bib)[greyBlock, right=of generator]{Client-Bibliothek\\\emph{PHP}};
	% Lines  
	\path[arrow, ->] (abstract) -- (parser);
	\path[arrow, ->] (parser) -- (generator);
	\path[arrow, ->] (templates) -- (generator);
    \path[arrow, ->] (infrastructurecode) -- (generator);
	\path[arrow, ->] (generator) -- (bib);
\end{tikzpicture}

    }
    \caption{Aufbau des Generatorsystems}
    \label{fig:generatorstructure}
\end{figure}        

Das Ziel dieser Arbeit ist die Erstellung eines Codegenerators, der aus der abstrakten Beschreibung der Spreadshirt-API eine Client-Bibliothek erstellt.

Der Generator soll eine flexible Wahl der Zielsprache bieten, wobei mit \enquote{Zielsprache} im folgenden die Programmiersprache der erzeugten Bibliothek gemeint ist. 
Für das Bibliotheksdesign ist eine \gls{DSL} (Domain-Specific Language) zu realisieren, mit dem Ziel die Nutzung der \gls{API} zu vereinfachen. 

Als Programmiersprache für den Generator wird \emph{Java} verwendet, als Zielsprache der Bibliothek dient \emph{PHP}. 
%Todo: Template-Engine, falls nicht, entfernen
\sout{Um die gewünschte Flexibilität bezüglich der Zielsprache zu erreichen, wird eine \gls{template-engine} verwendet.}
Eine gute Lesbarkeit, hohe Testabdeckung und größtmögliche Typsicherheit, soweit \emph{PHP} dies zulässt, sind Erfolgskriterien für die zu generierende Bibliothek.

\cref{fig:generatorstructure} stellt den schematischen Aufbau des gewünschten Generators dar.


%\section{Was ist Spreadshirt?}
%\label{sec:spreadshirt}

\begin{figure*}[!htb]
	\centering
		\includegraphics[width=2cm]{resources/sprd_logo_cmyk}
	\caption{Spreadshirt Logo}
	\label{fig:spreadshirtLogo}
\end{figure*}

Spreadshirt ist eines der führenden Unternehmen für personalisierte Kleidung und zählt zu den \emph{Social Commerce}-Unternehmen. Dieser Begriff beschreibt Handelsunternehmen bei denen die aktive Beteiligung und die persönliche Beziehung sowie Kommunikation der Kunden untereinander im Vordergrund stehen. 
Spreadshirt hat Standorte in Europa und Nordamerika, der Hauptsitz ist in Leipzig. 

Dem Nutzer wird eine Online-Plattform geboten um Kleidungsstücke selber zu gestalten oder zu kaufen, aber auch um eigene Designs, als Motiv oder in Form von Produkten, zum Verkauf anzubieten. 
Zusätzlich wird jedem Nutzer ermöglicht einen eigenen Shop auf der Plattform zu eröffnen und diesen auf der eigenen Internetseite einzubinden. Derzeit gibt es rund $\numprint{400000}$ 
Spreadshirt-Shops mit ca. 33 Millionen %$\numprint{33000000}$ 
Produkten.
Für die Spreadshirt-\gls{API} können Kunden eigene Anwendungen schreiben, beispielsweise \citetitle{zufallsshirt} \cite{zufallsshirt} oder \citetitle{soundslikecotton} \cite{soundslikecotton}.
Spreadshirt bedient neben dem Endkunden- auch das Großkundengeschäft als Anbieter von Druckleistungen.
% ToDo: verbessern des Abschnittes über Bulkbusiness

Die \emph{sprd.net AG}, zu der auch der Leipziger Hauptsitz gehört, beschäftigt derzeit 450 Mitarbeiter, davon 50 in der IT. %(Juni 2013) 178 Mitarbeiter, davon 29 in der \textsc{It}.

%\section{Motivation}
%\label{sec:motivation}

Die zwei wichtigsten Konstanten in der Anwendungsentwicklung sind laut \parencite{herrington2003code} folgende:
\begin{compactitem}
    \item Die Zeit eines Programmierers ist kostbar
    \item Programmierer mögen keine langweiligen und repetitiven Aufgaben
\end{compactitem}
Codegenerierung greift bei beiden Punkten an und kann zu einer Steigerung der \emph{Produktivität} führen, die durch herkömmliches schreiben von Code nicht zu erreichen wäre. 

Änderungen können an zentraler Stelle vorgenommen und durch die Generierung automatisch in den Code übertragen werden, was mit verbesserter \emph{Wartbarkeit} und erhöhter Effizienz einhergeht.
Die gewonnenen Freiräume kann der Entwickler nutzen um sich mit den Grundlegenden Herausforderungen und Problemen seiner Software zu beschäftigen.

Durch die Festlegung eines Schemas für Variablennamen und Funktionssignaturen, wird eine hohe \emph{Konsistenz}, über die gesamte Codebasis hinweg, erreicht.
Diese Einheitlichkeit vereinfacht auch die Nutzung des Generats\footnote{Ergebnis des Codegenerierungsvorgangs}, da beispielsweise nicht mit Überraschungen bei den verwendeten Bezeichnern zu rechnen ist.

% Als Eingabe für den Generator dient ein \emph{abstraktes Modell} des betreffenden Geschäftsbereiches. Die Erstellung eines solchen Modells vertieft das Verständnis des Entwicklers für das Geschäftsfeld und gibt gleichzeitig Spezialisten aus dem Fachbereich die Möglichkeit Fragestellungen anhand dieses Modells zu formulieren.

%Um die immer kürzer werdenden Entwicklungszyklen einhalten zu können, kann durch Codegenerierung die nötige Effizienzsteigerung geleistet werden.

Zusätzlich zu den bereits genannten allgemeinen Nutzen einer Codegenerierungslösung, entstehen für Spreadshirt noch die folgenden Vorteile:
\begin{compactitem}
    \item Vereinheitlichung bestehender Implementierungen in Form der generierten Bibliothek
    \item Vereinfachen der Authentifizierung durch Integration in Client-Bibliothek (siehe \cref{sec:api_auth})
    \item Erleichterung der \textsc{Api}-Nutzung für externe Entwickler
\end{compactitem}

% todo: allgemeine Motivation warum Codegenerierung ueberhaupt gemacht wird. Hier sollte aber speziell stehen warum gerade bei Spreadshirt Codegenerierung verwendet werden soll. Außerdem sollte anhand eines kurzen Beispiels gezeigt werden wie der Stand vor- und nach der Generierung war.

\section{Anforderungen an die Client-Bibliothek}
\label{item:requirements}

\begin{compactitem}
    \item Austauschbarkeit der Zielsprache
    \item Vollständige Generierung aller Methoden aus der \textsc{Api}-Beschreibung
    \item Einfache Bedienbarkeit
\end{compactitem}