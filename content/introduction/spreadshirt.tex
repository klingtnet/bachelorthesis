\section{Was ist Spreadshirt?}

\begin{figure*}[!htb]
	\centering
		\includegraphics[width=2cm]{resources/sprd_logo_cmyk}
	\caption{Spreadshirt Logo}
	\label{fig:spreadshirtLogo}
\end{figure*}

Spreadshirt ist eines der führenden Unternehmen für personalisierte Kleidung und zählt zu den \emph{Social Commerce}\footnote{Handelsunternehmen bei dem die aktive Beteiligung und persönliche Beziehung, sowie Kommunikation der Kunden untereinander, im Vordergrund stehen.}-Unternehmen. Es gibt Standorte in Europa und Nordamerika, der Hauptsitz ist in Leipzig. 
Den Nutzern wird eine Online-Plattform geboten um Kleidungsstücke selber zu gestalten oder zu kaufen, oder auch um eigene Designs, als Motiv oder in Form von Produkten, zum Verkauf anzubieten. 
Es wird jedem Nutzer ermöglicht einen eigenen Shop auf der Plattform zu eröffnen und ihn auf der eigenen Internetseite einzubinden. Derzeit gibt es rund $\numprint{400000}$ Spreadshirt-Shops mit ca. $\numprint{33000000}$ Produkten.
Für die Spreadshirt-API können Kunden eigene Anwendungen schreiben, bspw. \printhref{http://zufallsshirt.de/}{zufallsshirt.de} oder \printhref{http://www.soundslikecotton.com/}{soundslikecotton.com}.
Neben dem Endkunden- bedient Spreadshirt auch das Großkundengeschäft als Anbieter von Druckleistungen.
% ToDo: verbessern des Abschnittes über Bulkbusiness

Die \emph{sprd.net AG} zu der auch der Leipziger Hauptsitz gehört beschäftigt derzeit\footnote{Stand Juni 2013} 178 Mitarbeiter, davon 29 in der IT.