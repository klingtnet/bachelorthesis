%\section{Was ist Spreadshirt?}
%\label{sec:spreadshirt}

\begin{figure*}[!htb]
	\centering
		\includegraphics[width=2cm]{resources/sprd_logo_cmyk}
	\caption{Spreadshirt Logo}
	\label{fig:spreadshirtLogo}
\end{figure*}

Spreadshirt ist eines der führenden Unternehmen für personalisierte Kleidung und zählt zu den \emph{Social Commerce}-Unternehmen. Dieser Begriff beschreibt Handelsunternehmen bei denen die aktive Beteiligung und die persönliche Beziehung sowie Kommunikation der Kunden untereinander im Vordergrund stehen. 
Spreadshirt hat Standorte in Europa und Nordamerika, der Hauptsitz ist in Leipzig. 

Dem Nutzer wird eine Online-Plattform geboten um Kleidungsstücke selber zu gestalten oder zu kaufen, aber auch um eigene Designs, als Motiv oder in Form von Produkten, zum Verkauf anzubieten. 
Zusätzlich wird jedem Nutzer ermöglicht einen eigenen Shop auf der Plattform zu eröffnen und diesen auf der eigenen Internetseite einzubinden. Derzeit gibt es rund $\numprint{400000}$ 
Spreadshirt-Shops mit ca. 33 Millionen %$\numprint{33000000}$ 
Produkten.
Für die Spreadshirt-\gls{API} können Kunden eigene Anwendungen schreiben, beispielsweise \citetitle{zufallsshirt} \cite{zufallsshirt} oder \citetitle{soundslikecotton} \cite{soundslikecotton}.
Spreadshirt bedient neben dem Endkunden- auch das Großkundengeschäft als Anbieter von Druckleistungen.
% ToDo: verbessern des Abschnittes über Bulkbusiness

Die \emph{sprd.net AG}, zu der auch der Leipziger Hauptsitz gehört, beschäftigt derzeit 450 Mitarbeiter, davon 50 in der IT. %(Juni 2013) 178 Mitarbeiter, davon 29 in der \textsc{It}.