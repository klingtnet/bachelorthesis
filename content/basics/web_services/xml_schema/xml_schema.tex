% ToDo: Titel in Schemasprachen ändern?
\section{XML Schemabeschreibungssprachen (XML Schema)}
\label{sec:xmlschema}

\emph{XML Schema} bezeichnet XML-basierte Sprachen mit denen sich Elemente, Attribute und Aufbau eines XML-Dokumentes beschreiben lassen. 
Ein XML-Dokument wird als \emph{valid/gültig} gegenüber einem Schema bezeichnet, falls die Elemente und Attribute dieses Dokumentes die Bedingungen des Schemas erfüllen \cite{taxonomyXMLSchema}.
Neben XSD (siehe \cref{sec:xsd}) und RelaxNG (siehe \cref{sec:relaxng}) existieren noch weitere Schemasprachen, die hier aber aufgrund ihrer geringen Relevanz nicht behandelt werden. Die beiden hier behandelten Schemasprachen bieten den Vorteil selbst XML-Dokumente zu sein, somit können sie durch herkömmliche XML-Tools bearbeitet werden.

\subsection{XML Schema Description (XSD)}
\label{sec:xsd}

\textsc{XML} Schema Description ist ein stark erweiterte Nachfolger der \gls{DTD} (Document Type Definition), derzeit spezifiert in Version 1.1 \cite{XMLSchema11Specification}. 
Die Syntax von \emph{\gls{XSD}} ist \gls{XML}. Damit ist die Schemabeschreibung ebenfalls ein gültiges \gls{XML}-Dokument. Als Dateiendung wird üblicherweise \texttt{.xsd} verwendet.

Die Hauptmerkmale von \gls{XSD} sind nach \cite[Kapitel 3.2][]{taxonomyXMLSchema} die folgenden:
\begin{compactitem}
    \item komplexe Typen (strukturierter Inhalt)
    \item anonyme Typen (besitzen kein \texttt{type}-Attribut)
    \item Modellgruppen
    \item Ableitung durch Erweiterung oder Einschränkung (\enquote{derivation by extension/restriction})
    \item Definition von abstrakten Typen
    \item Integritätsbedingungen (\enquote{integrity constraints}):\\
        \emph{unique}, \emph{keys} und \emph{keyref}, dies entspricht den \emph{unique-}, \emph{primary-} und \emph{foreign}-keys aus dem Bereich der Datenbanken        
\end{compactitem}

\begin{minipage}{\textwidth}
\begin{lstlisting}[
    language=XML,
    caption=Beginn der \gls{XSD}-Datei für die Spreadshirt-\gls{API},
    label=lst:xsdIntro
    ]
<?xml version="1.0" encoding="UTF-8" standalone="yes"?> //@\ding{202}@//
<xs:schema xmlns:xs="http://www.w3.org/2001/XMLSchema"  targetNamespace="http://api.spreadshirt.net" version="1.0" elementFormDefault="qualified"> //@\ding{203}@//
    <xs:import namespace="http://www.w3.org/1999/xlink" schemaLocation="xlink.xsd"/> //@\ding{204}@//
    ...
\end{lstlisting}
\end{minipage}

Eine \gls{XSD}-Datei beginnt wie jede \gls{XML}-Datei mit der \gls{XML}-Deklaration \ding{202}.

Das Wurzelelement der Schemadefinition zeigt \ding{203}. 
Das Attribut \emph{xmlns:xs= "http://www.w3.org/2001/XMLSchema"} führt den Namespace-Prefix \emph{xs} ein und gibt außerdem an, dass die Elemente und vordefinierten Datentypen (\cref{fig:xsddatatypes}) aus dem Namensraum \emph{http://www.w3.org/2001/XMLSchema} verwendet werden. Durch das Attribut \emph{targetNamespace} wird der Namensraum der Elemente festgelegt, die in dieser Schemadefinition definiert werden. \emph{Version} gibt die \gls{XSD}-Version an.
Der Wert des Attributs \emph{elementFormDefault} gibt an, ob Elemente des Schemas den \emph{targetNamespace} explizit angeben müssen (\enquote{qualified}) oder ob dies implizit geschieht (\enquote{unqualified}), die Angabe ist optional.

Externe Schemadefinitionen lassen sich unter Angabe des Namensraumes und einer \gls{URI} zu der \gls{XSD}-Datei einbinden \ding{204}.

\gls{XML}-Schema Description erlaubt die Definition von simplen Typen (\enquote{SimpleType}) und Typen mit strukturiertem Inhalt (\enquote{ComplexType}).

\begin{lstlisting}[
    language=XML,
    caption=Beispiel für einen SimpleType namens \enquote{unit} der Spreadshirt-API,
    label=lst:xsdExampleUnit
]
<xs:simpleType name="unit">
    <xs:restriction base="xs:string"> //@\ding{202}@//
        <xs:enumeration value="mm"/> //@\ding{203}@//
        <xs:enumeration value="px"/> //@\ding{203}@//
    </xs:restriction>
</xs:simpleType>
\end{lstlisting}

\emph{SimpleType}-Definitionen dienen zur Beschreibung einfacher Typen wie \emph{Enumeratoren}, oder \emph{Listen} für Daten eines primitiven Typs. Ein Beispiel für die Definition eines Enumerators durch einen SimpleType zeigt \Cref{lst:xsdExampleUnit}. Der Basisdatentyp des Enumerators wird dabei durch die Angabe des Attributs \emph{base} \ding{202} festgelegt. Zuordnung von Werten zu dem Enumerator zeigt \ding{203}.

Durch einen SimpleType definierte Listen sind durch Leerzeichen separierte Strings, sie werden meist für den Wert eines Attributes einer \gls{XML}-Datei verwendet. 

\begin{minipage}{\textwidth}
\begin{lstlisting}[
    language=XML,
    caption=Beispiel für einen Listentyp definiert duch einen SimpleType,
    label=lst:xsdListExample    
]
<xs:simpleType name=colors>
    <xs:list itemType="xs:string"/>
</xs:simpleType>
\end{lstlisting}
\end{minipage}

\begin{lstlisting}[
    language=XML,
    caption=Beispielinstanz für Typ aus \Cref{lst:xsdListExample},
    label=lst:xsdListExampleInstance
]
<test>red green blue</test>}
\end{lstlisting}

Die Definition eines strukturierten Typs zeigt \Cref{lst:xsdExampleAbstractList}.

\begin{lstlisting}[
    language=XML, 
    caption=Beispiel für eine Schemabeschreibung mit \gls{XSD} anhand des \enquote{abstractList}-Typs der Spreadshirt-API,
    label=lst:xsdExampleAbstractList
    ]
<xs:complexType name="abstractList" abstract="true"> //@\ding{202}@//
    <xs:sequence> //@\ding{203}@//
        <xs:element minOccurs="0" //@\ding{204}@// name="facets"> //@\ding{205}@//
            <xs:complexType> //@\ding{206}@//
                <xs:sequence>
                    <xs:element xmlns:tns="http://api.spreadshirt.net" minOccurs="0" maxOccurs="unbounded" ref="tns:facet" //@\ding{207}@// />
                </xs:sequence>
            </xs:complexType>
        </xs:element>
    </xs:sequence>
    <xs:attribute xmlns:xlink="http://www.w3.org/1999/xlink" ref="xlink:href"/> //@\ding{208}@//
    <xs:attribute type="xs:long" name="offset"/> //@\ding{209}@//
    <xs:attribute type="xs:string" name="query"/>        
    ...
</xs:complexType>
\end{lstlisting}

Das \emph{ComplexType}-Tag \ding{202} umschließt die Definiton des strukturierten Typs. \gls{XML}-Schema Description erlaubt das Definieren von abstrakten Typen, nur Ableitungen davon dürfen als Instanzen in einem Dokument auftreten. Abgeleitete Typen dürfen dabei den abstrakten Typ \emph{erweitern} oder \emph{einschränken} (\enquote{derivation by extension/restriction}).

Mit \emph{Reihenfolgeindikatoren} \ding{203} kann die Ordnung von Elementen festgelegt werden. Elemente unterhalb eines \emph{Sequence}-Tags dürfen nur in der Abfolge auftreten, in der sie definiert worden sind. Das \emph{All}-Tag \ding{205} hingegen erlaubt das Vorkommen ohne festgelegte Reihenfolge. Der Reihenfolgeindikator \emph{Choice} erlaubt nur eines der Elemente, die unterhalb dieses Tags vorkommen.

Durch die optionale Angabe von \emph{Häufigkeitsindikatoren} \ding{204} kann festgelegt werden wie oft ein Element an der definierten Stelle vorkommen darf. Entfällt dies, entspricht der Wert von \emph{minOccurs} "1" und \emph{maxOccurs} "1", das heißt, das Element darf genau einmal an dieser Stelle vorkommen.

Elemente einer \gls{XML}-Datei werden durch das gleichnamige \emph{Element} \ding{205} im \gls{XSD} definiert. Ein Element benötigt die Angabe eines Namens und Typs. Die Angabe des Typs kann dabei als Referenz auf die Typdefinition \ding{207} oder als Definition unterhalb des Element-Tags erfolgen \ding{206}.

Attribute eines \gls{XML}-Tags werden durch das \emph{Attribute}-Element definiert. Dies geschieht durch Angabe von Name und Typ \ding{208} oder durch eine Referenz auf eine Attributdefinition \ding{207}.

Referenzen haben die Form \emph{Namensraumbezeicher}:\emph{Elementname}. Wobei mit Elementname jedes Element der Schemabeschreibung gemeint ist, welches ein \emph{name}-Attribut besitzt. Der konkrete Namensraum eines solchen Bezeichners wird vorher mit der Angabe eines Attributes in dieser Form eingeführt:

\[  
    \underbrace{\text{\texttt{xmlns}}}_{\text{\ding{202}}}
    :
    \underbrace{\text{\texttt{tns}}}^{\text{\ding{203}}}
    \texttt{=}
    \underbrace{
        \texttt{"http://api.spreadshirt.net"}
    }_{\text{\ding{204}}}
\]

\begin{compactitem}
    \item[\ding{202}] \gls{XML}-Namespace
    \item[\ding{203}] Namensraumbezeichner
    \item[\ding{204}] Konkreter Namensraum
\end{compactitem}

%
% todo: xml schema genauer beschreiben, relaxng entfernen --: Beispiel fuer Instanz und Definition des Typs!
%

\begin{sidewaysfigure}
    \centering
    \tikzstyle{blueBox}=[
        rectangle,
        fill={blue!15},
        draw,
        font=\sffamily
    ]      
    \tikzstyle{grayBox}=[
        rectangle,
        fill=lightgray,
        text=black,
        font=\sffamily,
        draw
    ]
    \tikzstyle{violetBox}=[
        rectangle,
        fill=violet,
        text=white,
        font=\sffamily,
        draw
    ]
    \tikzstyle{greenBox}=[
        rectangle,
        fill=green!50,
        text=black,
        font=\sffamily,
        draw
    ]
    \tikzstyle{derivedFromList}=[
        dashed,
        cyan
    ]
    \resizebox{\textheight}{!}{
            \begin{tikzpicture}[
    level distance=1.1cm,
    level 1/.style={sibling distance=4cm},
    level 2/.style={sibling distance=2cm},
    level 3/.style={sibling distance=2.5cm}
  ]
  \node (root) [violetBox] {anyType}
    [edge from parent fork down]
    child {node[grayBox] {all complex types}
        edge from parent[loosely dashed, magenta]
    }
    child {node[violetBox] {anySimpleType}
            child {node[blueBox] {duration}}
            child {node[blueBox] {dateTime}}
            child {node[blueBox] {time}}
            child {node[blueBox] {date}}
            child {node[blueBox] {gYearMonth}}
            child {node[blueBox] {gYear}}
            child {node[blueBox] {gMonthDay}}
            child {node[blueBox] {gDay}}
            child {node[blueBox] {gMonth}}
            child {
                child [sibling distance = 3cm]{
                    child {node[blueBox] {string}
                        child {node[greenBox] {normalizedString}}
                        child {node[greenBox] {token}
                            child {node[greenBox] {language}}
                            child {node[greenBox] {Name}
                                child {node[greenBox] {NCName}
                                    child {node[greenBox] {ID}}
                                    child {node[greenBox] {IDREF}
                                        child {node[greenBox] {IDREFS}
                                            edge from parent[derivedFromList]
                                        }
                                    }
                                    child {node[greenBox] {ENTITY}
                                        child {node[greenBox] {ENTITIES}
                                            edge from parent[derivedFromList]
                                        }
                                    }
                                }
                            }
                            child {node[greenBox] {NMTOKEN}
                                child  {node [greenBox] {NMTOKENS}
                                    edge from parent[derivedFromList]
                                }
                            }
                        }
                    }
                }
                child {node[blueBox] {boolean}}
                child {node[blueBox] {base64Binary}}
                child {node[blueBox] {hexBinary}}
                child {node[blueBox] {float}}
                child [sibling distance = 3cm] {
                    child {node[blueBox] {decimal}
                        child [sibling distance = 4cm] {node[greenBox] {integer}
                            child {node[greenBox] {nonPositiveInteger}
                                child {node[greenBox] {negativeInteger}}
                            }
                            child {node[greenBox] {long}
                                child {node[greenBox] {int}
                                    child {node[greenBox] {short}
                                        child {node[greenBox] {byte}}
                                    }
                                }
                            }
                            child {node[greenBox] {nonNegativeInteger}
                                child {node[greenBox] {unsignedLong}
                                    child {node[greenBox] {unsignedInt}
                                        child {node[greenBox] {unsignedShort}
                                            child {node[greenBox] {unsignedByte}}
                                        }
                                    }
                                }
                                child {node[greenBox] {positiveInteger}}
                            }
                        }
                    }
                }
                child {node[blueBox] {double}}
                child {node[blueBox] {anyURI}}
                child {node[blueBox] {QName}}
                child {node[blueBox] {NOTATION}}
            }
    };
\end{tikzpicture}
    }
    \vspace{\baselineskip}\\
    \resizebox{0.5\textheight}{!}{
            \begin{tikzpicture}[framed]
    \node (title) [font=\bfseries] {Legende:};
    \node (base) [violetBox, right = of title] {Basis Typ};
    \node (primitive) [blueBox, right = of base] {Primitiver Typ};
    \node (derived) [greenBox, right = of primitive] {Abgeleiteter Typ};
    \node (complex) [grayBox, right = of derived] {Komplexer Typ};
    \node (d1) [below = of base] {};
    \node (d2) [below = of primitive] {}
        edge [] node[swap, align=center]{Abgeleitet durch\\Einschränkung} (d1);
    \node (d3) [below = of derived] {};
    \node (d4) [below = of complex] {}
        edge [dashed, cyan] node[swap, align=center]{von Liste\\abgeleitet} (d3);
    \node (d5) [below = of d2] {};
    \node (d6) [below = of d3] {}
        edge [loosely dashed, magenta] node[swap, align=center]{Abgeleitet durch\\Erweiterung/Einschränkung} (d5);
\end{tikzpicture}
    }        
    \caption{vordefinierte \gls{XSD} Datentypen nach \cite{XMLSchema11Specification} Kapitel 3}
    \label{fig:xsddatatypes}
\end{sidewaysfigure}

\subsection{RelaxNG}
\label{sec:relaxng}

% Wikipedia, satz abändern
Ebenso wie XSD (\cref{sec:xsd}) ist die \emph{Regular Language Description for XML New Generation} eine XML-Schemasprache zur Definition der Struktur von XML-Dokumenten. 
Schemas werden in \emph{RelaxNG} durch XML-Syntax oder eine eigene, kompaktere nicht-XML Syntax formuliert. Ebenso wie bei \emph{XML Schema} werden Namespaces unterstützt. RelaxNG Schemabeschreibungen verwenden meist \texttt{.rng} als Dateiendung.

Unterschiede zu XML Schema:
\begin{compactitem}
    \item Unterstützung von ungeordneten Inhalten
    \item kompaktere nicht-XML Syntax
    \item \emph{nichtdeterministisches} oder auch \emph{mehrdeutiges} Inhaltsmodell \cite[Kapitel 16]{RelaxNGVlist}    
    % Beispiel von http://pike.psu.edu/publications/toit05.pdf Seite 17
\end{compactitem}

\begin{lstlisting}[
    language=XML, 
    caption=Minimalbeispiel für eine Schemadefinition in RelaxNG, 
    label=minimalRelaxNG]
<?xml version="1.0" encoding="utf-8"?>
<grammar 
    xmlns="http://relaxng.org/ns/structure/1.0"
    ns="myNamespace">
    <start>
        <ref name="product"/>
    </start>
    <define name="product">
        <oneOrMore>
            <element name="name"/>
                <text/>
            </element>
            <element name="price"/>
                <text/>
            </element>
            <ref name="description"/>
        </oneOrMore>
    </define>
    <define name="description">
        <oneOrMore>
            <element name="title">
                <text/>
            </element>
            <element name="content">
                <text/>
            </element>
        </oneOrMore>
    </define>
</grammar>
\end{lstlisting}
