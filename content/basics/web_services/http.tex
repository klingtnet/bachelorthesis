\section{HTTP}
\label{sec:http}

\thesisDefinition{\gls{HTTP}}{
    Das \emph{Hypertext Transfer Protocol (\gls{HTTP})} ist ein allgemeines und \emph{zustandsloses} Protokoll, zur Übertragung von Daten über ein Netzwerk, was durch Erweiterung seiner Anfragemethoden, Statuscodes und Header für viele unterschiedliche Anwendungen verwendet werden kann (\cite[][Abstract]{rfc2616}).
}

\gls{HTTP} arbeitet auf der Anwendungsschicht des \gls{OSI} und ist somit unabhängig von dem zum Datentransport eingesetzten Protokoll. Über eindeutige \glspl{URI} werden \gls{HTTP}-Ressourcen angesprochen. Dabei sendet ein \emph{Client} eine Anfrage (\emph{request}) und erhält daraufhin vom Server eine Antwort (\emph{response}). Anfrage und Antwort stellt dabei eine \gls{HTTP}-Nachricht dar, die aus den zwei Elementen \emph{Header} und \emph{Body} besteht. Letzterer trägt die Nutzdaten und kann, je nach verwendeter \gls{HTTP}-Methode, auch leer sein.

\subsection{Methoden}
\label{sec:http_methods}

\citetitle{rfc2616} (\cite{rfc2616}) definiert einige \gls{HTTP}-Methoden, wobei die am meisten verwendeten die folgenden sind:
\begin{compactitem}
    \item \textsc{Get} (Aufruf einer Webseite im Browser $\rightarrow$ \textsc{Get}-Request an Server)
    \item \textsc{Put}
    \item \textsc{Post} (Übermittlung von Formulardaten eines Client an Server)
    \item \textsc{Delete}
    \item \textsc{Options}
\end{compactitem}

Alle bis auf \textsc{Post} und \textsc{Options} sind \emph{idempotent} (\cite{rfc2616} Kapitel 9), d.h. eine Komposition (Ausführung der Methoden hintereinander) der Methode führt zu demselben Ergebnis wie ein einzelner Aufruf.

Methodennamen sind üblicherweise Verben die die auszuführende Aktion beschreiben, deswegen werden HTTP-Methoden auch \enquote{verbs} genannt. Die Definition von eigenen Methoden ist möglich.

%Sie gehören zur Gruppe der sogenannten \emph{CRUD}-Methoden (Create, Receive, Update, Delete) .
%\emph{Header} und \emph{Body} bilden die grundlegenden Elemente einer HTTP-Nachricht. 

\subsection{Header}
\label{sec:http-header}

Ein Header einer \gls{HTTP}-Nachricht besteht aus einer \emph{Request Line} (erste Zeile des Headers) und einer Menge von Schlüssel-Wert-Paaren. \Cref{lst:headGETrequest} zeigt einen Beispiel-Header für die \textsc{Get}-Anfrage auf die Spreadshirt-\gls{API} Ressource:
\texttt{http://api.spreadshirt.net/api/v1/locales}.

\begin{lstlisting}[
    label=lst:headGETrequest,
    caption=\gls{HTTP}-Header von \textsc{Get} Request auf Spreadshirt-\gls{API} Ressource \texttt{http://api.spreadshirt.net/api/v1/locales}
]
GET //@\ding{202}@// /api/v1/locales //@\ding{203}@// HTTP/1.1 //@\ding{204}@//
User-Agent: curl/7.29.0 //@\ding{205}@//
Host: api.spreadshirt.net //@\ding{206}@//
Accept: */* //@\ding{207}@//
\end{lstlisting} 

\begin{compactitem}
    \item[\ding{202}] Angabe der \gls{HTTP}-Methode
    \item[\ding{203}] Ressource
    \item[\ding{204}] verwendete \gls{HTTP}-Version
    \item[\ding{205}] \emph{User-Agent}, Angabe zum Client-System, das die Anfrage versendet
    \item[\ding{206}] \emph{Host}, der Server welcher die Anfrage erhält und der die Ressource \ding{203} verwaltet
    \item[\ding{207}] Angabe von \emph{Content-Types}, die der Client als Antwort akzeptiert, in diesem Fall eine \emph{Wildcard}, also alle Typen sind als Antwort erlaubt
\end{compactitem}

Nachfolgend die \emph{Response} auf den soeben beschrieben \emph{Request}.

\begin{lstlisting}[
    label=lst:headGETresponse,
    caption=\gls{HTTP}-Header von \textsc{Get} Response aus Spreadshirt-\gls{API} Ressource \texttt{http://api.spreadshirt.net/api/v1/locales}
]
HTTP/1.1 200 OK //@\ding{202}@//
Expires: Tue, 20 Aug 2013 19:05:25 GMT
Content-Language: en-gb
Content-Type: application/xml; //@\ding{203}@//
charset=UTF-8 //@\ding{204}@//
X-Cache-Lookup: MISS from fish07:80
X-Server-Name: mem1
True-Client-IP: 88.79.226.66
Date: Tue, 20 Aug 2013 07:20:25 GMT
Content-Length: 1659
Connection: keep-alive
\end{lstlisting}

\begin{compactitem}
    \item[\ding{202}] \emph{Response Line}, Angabe der \gls{HTTP}-Version am Anfang und danach der \gls{HTTP}-Statuscode mit Kurzbeschreibung
    \item[\ding{203}] Angabe des Content-Types des Bodys der Nachricht (hier \gls{XML})
    \item[\ding{204}] Kodierung der Nachricht
\end{compactitem}

Welche Einträge der Header einer \gls{HTTP}-Nachricht letztendlich enthält, ist abhängig von der Implementierung des Clients oder Servers und es können auch jederzeit eigene Einträge, die nicht in der \gls{HTTP}-Spezifikation enthalten sind, hinzugefügt werden.

\subsubsection{Authorization Request Header}
\label{sec:http-authorization-header}

Die \gls{HTTP}-Spezifikation \cite[][Abschnitt 14]{rfc2616} sieht ein Feld für Autorisierungsinformationen im Header vor. 
%Diese Feld wird meist von einem Client gesetzt um sich an einem Web Service zu autorisieren. 
Der Feld ist folgendermaßen aufgebaut: \texttt{Authorization: credentials}. Der Aufbau des Berechtigungsnachweises (engl. \enquote {credentials}) ist nicht näher spezifiziert und kann vom Web Service Betreiber selbst festgelegt werden. Die Spreadshirt-\gls{API} definiert einen eigenen Autorisierungs-Header (\cref{sec:api_auth}).

\subsection{Body}
\label{sec:http-body}

\begin{minipage}{\textwidth}
\begin{lstlisting}[
    language=XML,
    label=lst:bodyGETresponse,
    caption=\gls{HTTP}-Body der Response aus der \textsc{Get}-Methode auf der Spreadshirt-\gls{API}-Ressource \texttt{http://api.spreadshirt.net/api/v1/locales}
]
<?xml version="1.0" encoding="UTF-8" standalone="yes"?>
<locales 
    xmlns:xlink="http://www.w3.org/1999/xlink" 
    xmlns="http://api.spreadshirt.net" 
    xlink:href="http://api.spreadshirt.net/api/v1/locales" 
    offset="0" limit="50" count="16" 
    sortField="default" sortOrder="default">
    <locale 
        xlink:href="http://api.spreadshirt.net/api/v1/locales/de_DE" 
        id="de_DE"/>
    ...
    <locale 
        xlink:href="http://api.spreadshirt.net/api/v1/locales/nl_BE" 
        id="nl_BE"/>
</locales>
\end{lstlisting}
\end{minipage}

Der Body enthält die eigentlichen Nutzdaten. Deren Format wird mit dem \emph{Content-Type}-Eintrag des Headers angegeben. \Cref{lst:bodyGETresponse} zeigt den Body der \emph{response} von \Cref{lst:headGETresponse}.
