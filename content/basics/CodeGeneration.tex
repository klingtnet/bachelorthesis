\section{Codegeneratoren}
\label{sec:codegenerators}

%Generative Programming p. 333
Ein \emph{Codegenerator} ist ein Programm, welches aus der höhersprachigen Spezifikation\footnote{mit anderen Worten: auf einem höheren Abstraktionslevel als die zur Implementierung verwendete Programmiersprache} einer Software oder eines Teilaspektes, die Implementierung erzeugt (nach \cite{czarnecki2000generative}).

% todo: folgende Liste entfernen?
Generatoren widmen sich drei wichtigen Problemen\cite{czarnecki2000generative}:
\begin{description}
    \item[Relevanz von Systembeschreibungen erhöhen] Eine Systembeschreibung sollte direkt und explizit die Anforderungen bestimmen und mit der Sprache der Problemdomäne formuliert sein.
    \item[Erzeugung einer effizienten Implementierung] Die größte Herausforderung bei der Erstellung eines Generators liegt in der Abbildung von der Spezifikation zur Implementierung, da es meist keine direkte Übereinstimmung zwischen beiden Konzepten gibt.
    \item[\enquote{Library scaling problem}] Nur die durch die Spezifikation benötigten Methoden generieren.
\end{description}

