\section{Konzeptuelles Modell}

Ein Modell bildet die Funktionen und Beziehungen eines Bereiches der Wirklichkeit ab.
Ein solches Modell, beispielsweise in Form einer \gls{WADL}-Datei zur Beschreibung einer Web-API, dient als Eingabe für einen Generator. Außerdem ist es der Ausgangspunkt in der modell-getriebenen Softwareentwicklung (\gls{MDSD}) oder -architektur (\gls{MDA}).

\begin{figure}[tb]
    \centering
    %\resizebox{\textwidth}{!}{
        \begin{tikzpicture}[
                node distance=12mm and 8mm,
                every node/.style={font=\scriptsize}
            ]
            \node(model)[greyBlock]{Modell};
            \node(generator)[greyBlock, right=of model]{Codegenerator};
            \node(sourcecode)[greyBlock, double copy shadow, right=of generator]{Quellcode};
            \node(templates)[greyBlock, double copy shadow, above=of generator]{Templates};
            \node(infrastructurecode)[greyBlock, double copy shadow, below=of generator]{Infrastrukturcode};     

            \path [arrow, ->] (model) -- (generator);
            \path [arrow, ->] (templates) -- (generator);
            \path [arrow, ->] (infrastructurecode) -- (generator);
            \path [arrow, ->] (generator) -- (sourcecode);
        \end{tikzpicture}   
    %}
    \caption{Simples Generatorsystem}
    \label{fig:generatorsystem}
\end{figure}
