\section{Ausblick}
\label{sec:prospect}

% todo: implementierung anderer Sprachmodelle, Schnittstelle zu Generator

Der in dieser Arbeit dokumentierte Codegenerator bietet mehrer Ansatzpunkte für Erweiterungen oder Verbesserungen:

\begin{compactitem}
    \item Generierung eines \emph{Fluent-Interface} Pattern. Dieses Entwurfsmuster wurde von \citeauthor{fowler2010domain} in \cite{fowler2010domain} beschrieben und basiert auf der Technik des \enquote{method-chaining}, also der Hintereinanderausführung von Methoden, wobei jede Methode mit dem Resultat der vorangegangen arbeitet.
    \item Implementierung weiterer Sprachenmodelle, bspw. zur Generierung einer Java-Bibliothek. 
    \item Erzeugung von Tests durch den Generator um automatisch die generierte Bibliothek prüfen zu können.
\end{compactitem}